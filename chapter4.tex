\section{Hahn-Banach Theorem and Corollaries}
 % To be overlooked %
\begin{definition}
\begin{enumerate}
	\item[a)] Let $E$ be a linear space. Then the algebraic dual space of $E$ which is the space of linear maps $E \rightarrow \K$, is denoted $E'$.
	\item[b)] Let $E$ be an $\R$-vector space. Then $p: E \rightarrow \R$ is a sublinear functional on $E$ if all $x,y \in E$ and $\lambda \in \R$ satisfy
	$p(x+y) \le p(x) + p(y)$ and $p(\lambda x) = \lambda p(x)$,
\end{enumerate}
\end{definition}

\begin{lemma}
... to be continued
\end{lemma}

% Axel Flinth ----------------------------
\paragraph*{WARNING} The first lemma of the day uses the Lemma of Zorn. If you're not comfortable with it, then essentially everything in this lecture will be very hard to swallow!

When now everyone has been warned it is time for

\begin{lemma} Let $E$ be an $\R$-vector space, $F \subseteq E$ a linear subspace and $p$ a sublinear functional on E. Further let $f \in F'$ with
\begin{equation*}
f(x) \leq p(x) \quad \forall x \in F
\end{equation*}
Then there exists $l \in E'$ with $l\vert_F =f$ and $l(x) \leq p(x) \forall x \in E$.
\end{lemma}
\begin{proof}
Set
\begin{eqnarray*}
\mathcal{L}= \{ (L,l)&:& L \text{ linear subspace of } E \text{ with } L\supset F  \\
                     && \text{and } l \in L' \text{ with } l\vert_ F =f \text{ and } l(x)\leq p(x) \forall x \in L\}
\end{eqnarray*}
To prove is the existence of a pair of the form $(E,l) \in \L$. To do this, we define an ordering on $\L$.
\begin{equation*}
(L_1, l_1) \leq (L_2,l_2) \Leftrightarrow L_1 \subset L_2 \land l_2\vert_{L_1} = l_1
\end{equation*}
We know that $\mathcal{L}\neq \varnothing$, then we have $(F,f) \in \L$. 

We want to use the Lemma of Zorn. Let $\mathscr{K}$ be a chain in $\L$. To prove is that $\KK$ has an upper bound in $\L$. We claim that if we set
\begin{equation*}
\tilde{L}= \{ L : \exists l' \text{ with } ( L,l') \in \mathscr{K} \}
\end{equation*} 
then $\tilde{L}$ is a linear subspace since $\KK$ is linearly ordered. define $\tilde{l}: \tilde{L} \to \R$ through
\begin{eqnarray*}
\tilde{l}(x) := l(x) &&\text{if } x \in L \text { and }\\
 && l \in L' \text{ with } (L,l) \in \mathscr{K}
\end{eqnarray*}
We know by the definition of $\tilde{L}$ that such an $l$ always exists. In fact, this definition is well-defined, since if $(L_2,l_2) \in \KK$ is another pair with $x \in L_2$, then one of the pairs is bigger with respect to the ordering ($\mathscr{K}$ is a chain), WLOG $(L_1,l_1) \leq (L_2,l_2)$. Then, because of $x \in L_1$ we have $l_2(x)=l_1(x)$.

$\tilde{l}\in \tilde{L}'$. To prove is only the linearity. Let $x_i \in L_i, \alpha_i \in \mathbb{R}, (L_i, l_i) \in \mathscr{K}, i=1,2$. Then WLOG $(L_1,l_1) \leq (L_2,l_2)$. Therefore $x_1  \in L_2$ and thus $\alpha_1 x_1 +\alpha_2 x_2 \in  L_2$. There follows
\begin{eqnarray*}
\tilde{l}(\alpha_1 x_1 + \alpha_2 x_2) &=& l_2(\alpha_2 x_1 +\alpha_2 x_2)  = \alpha_1 l_2(x_1) + \alpha_2 l_2(x_2) \\
 &=& \alpha_1 l_1(x_1) + \alpha_2 l_2(x_2) = \alpha_1 \tilde{l}(x_1) + \alpha_2 \tilde{l}(x_2)
\end{eqnarray*}


Finally $\tilde{l}(x) =l(x) \leq p(x)$. The Lemma of Zorn now provides the existence of a maximal element of $\mathcal{L}$ $(L,l) \in \mathcal{L}$. To prove is $L=E$. 

Suppose the opposite, then there exists $x_0 \in E \backslash L$. By Lemma 4.2, there exists $g \in (L + \R x_0)'$ with $g \vert_L =l$ and $g(x)\leq p(x) \quad \forall x \in (L+\mathbb{R}x_0)$. Then we although have $(L,l) < (L + \mathbb{R}x_0, g)$. Contradiction!

Thus $L=E$, this proves the Lemma.
\end{proof}

\begin{theo} Let $E$ be a vector space over $\mathbb{K}$, $F$ a linear subspace and $f \in F'$.

Let $p: E \to \mathbb{R}$ be a seminorm on E, i.e. for all $x,y \in E, \lambda \in \mathbb{K}$
\begin{eqnarray*}
p(x+y) &\leq & p(x) + p(y) \\
p(\lambda x) &=& \vert \lambda \vert p(x)
\end{eqnarray*}

Suppose that $\vert f(x) \vert \leq p(x)$ for all $x \in F$. Then there exists an $l \in E'$ with $l\vert_F=f$ and $\vert l(x) \vert \leq p(x) \quad \forall x \in E$.
\end{theo}

\begin{proof}
First consider $\mathbb{K}=\mathbb{R}$. Then $f(x) \leq p(x)$ for all $x \in F$ and $p(\alpha x )= \alpha p(x)$ for all $x\in E , \alpha \geq 0$. By Lemma 4.3, there exists some $l \in E'$ with $l\vert_F=f$ and $l(x)\leq p(x) \quad \forall x \in E$. Since also
\begin{equation*}
-l(x) = l(-x) \leq p(-x)=p(x)
\end{equation*}
we have $\vert l(x) \vert \leq p(x)$.

The case $\mathbb{K}=\mathbb{C}$ will be discussed in the exercises.
\end{proof}

\begin{theo}{(Hahn-Banach Theorem)} Let $E$ be a normed space, $F$ a linear subspace of $E$. Then for each $f \in F^{\ast}$ there exists some $l \in E^{\ast}$ with
\begin{equation*}
l \vert_F =f \land \Vert l \Vert = \Vert f \Vert
\end{equation*}
\end{theo}

\begin{proof}
Let $p$ be defined by
\begin{equation*}
p(x) = \Vert f \Vert \Vert x \Vert
\end{equation*}
Then $p$ is a seminorm - the properties are inherited from the norm properties of $\Vert \cdot \Vert$. Furthermore
\begin{equation*}
\vert f(x) \vert \leq p(x) \quad \forall x\in F
\end{equation*}
By Theorem 4.4, there exists $l \in E'$ with $l \vert_F=f$ and $ \vert l (x) \vert \leq p(x)= \Vert f \Vert \Vert x \Vert$. This proves in particular that $l \in E^{\ast}$ and $\Vert l \Vert \leq \Vert f \Vert$. Because of $l \vert_F=f$, the reverse inequality holds. Thus $\Vert l \Vert = \Vert f \Vert$. 
\end{proof}

\begin{kor}
Let $E$ be a normed space and $F$ a linear subspace of $E$ and $x \in E$ such that
\begin{equation*}
\delta := \inf_{y \in F} \Vert x - y \Vert >0
\end{equation*}
Then there exists an $l \in E^{\ast}$ with
\begin{equation*}
l \vert_F =0, \Vert l \Vert =1 \text{ and } l(x)=\delta
\end{equation*}
In particular, for any $x\neq0$ there exists an $l \in E^{\ast}$ with $\Vert l \Vert =1$ and $l(x)= \Vert x\Vert $.
\end{kor}
\begin{proof}
Let $G=F + \mathbb{K}x$ and $g: G \to \mathbb{K}$ be defined through
\begin{equation*}
g(y + \lambda x)= \lambda \delta \quad \forall y \in F, \lambda \in \mathbb{K}
\end{equation*}
$g$ is well defined, then because of $x \notin F$ we have $G = F \oplus \mathbb{K}x$. Further $g$ is linear, $g\vert_F=0$ and $g(x)=\delta$.

We know claim $\Vert g \Vert =1$. There holds
\begin{eqnarray*}
\vert g(y + \lambda x) \vert = \vert \lambda \vert \delta &=& \vert \lambda \vert \inf_{z \in F} \Vert z-x \Vert \\ &=& \inf_{z \in F} \Vert \lambda z - \lambda x \Vert = \inf_{z \in F} \Vert z+\lambda x \Vert \leq \Vert y + \lambda x \Vert
\end{eqnarray*}
thus $\Vert g \Vert \leq 1$.

Secondly, there exists for every $\epsilon>0$ a $z_{\epsilon} \in F$ with $\delta \leq \Vert z_{\epsilon} + x \Vert \leq \delta + \epsilon$. There follows
\begin{eqnarray*}
g(x + z_{\epsilon})&=& \delta \geq \Vert x+z_{\epsilon} \Vert - \epsilon \Leftarrow \\
g(\Vert x+ z_{\epsilon} \Vert^{-1}(x+z_{\epsilon})) &=&  \frac{\delta}{\Vert x+z_{\epsilon}\Vert} \geq 1 - \frac{\epsilon}{\Vert x + z_{\epsilon}} \geq 1-\frac{\epsilon}{\delta}
\end{eqnarray*}
Now apply Theorem 4.5 to lift $g$ up to $E^{\ast}$.

For the in-particular part, choose $F=\{0\}$.
\end{proof}

We know define an important concept
\begin{definition}
Let $E$ be a normed space, $M \subset E$ an arbitrary subset of $E$ and $L \in E^{\ast}$ one of $E^{\ast}$. Then the annihilator of $M$ in $E^{\ast}$ is
\begin{equation*}
M^{\perp} := \{ l \in E^{\ast} : l(x) =0 \forall x \in M \}
\end{equation*}
and the annihilator of $L$ in $E$ is
\begin{equation*}
L_{\perp} = \{ x \in E: l(x)=0 \forall l\in L \}
\end{equation*}
\end{definition}

\begin{rem} 
The annihilators are closed linear subspaces of $E^{\ast}$ and $E$, respectively. This follows from the continuity of $l \mapsto l(x), x \mapsto l(x)$.
\end{rem}

\begin{lemma} Let $E$ be a normed space and $\varnothing \neq M \subseteq E$. Then $(M^{\perp})_{\perp}$ is the closed linear hull of $M$, i.e. the smallest closed linear subspace of $E$ which contains $M$.
\end{lemma}
\begin{proof}
If $x \in M$, then $l(x)=0$ for all $l \in M^{\perp}$, thus $x \in (M^{\perp})_{\perp} \Leftarrow M \subseteq (M^{\perp})_{\perp}$.

Now let $F$ be the closed linear hull of $M$. By the remark, $F \subseteq (M^{\perp})_{\perp}$. Now assume there exists $x \in (M^{\perp})_{\perp} \backslash F$. Corollary 4.6 secures the existence of an $l \in (M^{\perp})_{\perp}^{\ast}$ with $l \vert_F=0$ and $l(x)=0$.

Theorem 4.5 now implies the existence of an $f \in E^{\ast}$ with $f\vert_{(M^{\perp})_{\perp}}=l$. $f$ is in $M^{\perp}$ because of $f\vert_F = l\vert_F =0$ and $M\subseteq F$. But $f(x)\neq 0$ - Contradiction! 
\end{proof}

\setcounter{theo}{9}  %% Remove this when continuing the script!!!!!!!!!!!
\begin{theo}\label{4:t:isoiso}
Let $E$ be a normed space over $\K$, and $F\sse E$ a linear subspace.
\begin{enumerate}
\item[(i)] The linear operator
$$
\Phi : E^*/F^\perp\to F^*,\quad \Phi(f + F^\perp) = f|_F,\;f\in E^*,
$$
is an isometric isomorphism.
\item[(ii)] If $F$ is closed, then the linear operator
$$
\Phi : (E/F)^*\to F^\perp,\quad (\Phi f)(x) = f(x+F),\;x\in E,\,f\in (E/F)^*,
$$
is an isometric isomorphism.
\end{enumerate}
\end{theo}
\begin{proof}
(i). Consider the map $T : E^*\to F^*$, $Tf := f|_F$, $f\in E^*$. We have $\ker T = F^\perp$. Hence, $\Phi$ is well-defined, linear, and injective. By Theorem 4.5\marginpar{Reference!}, for each $\ell\in F^*$ there exists some $f\in E^*$ with $f|_F = \ell$. Hence, $\Phi$ is surjective. Finally, let $f\in E^*$, and choose $g\in E^*$ such that
$$
\|g\| = \|f|_F\|\quad\text{and}\quad g|_F = f|_F.
$$
Then, we obtain
$$
\|f + F^\perp\| \le \|f + (g-f)\| = \|g\| = \|f|_F\| = \|\Phi(f + F^\perp)\|.
$$
On the other hand, for all $g\in F^\perp$,
$$
\|\Phi(f + F^\perp)\| = \|f|_F\| = \|(f+g)|_F\|\le \|f + g\|.
$$
Hence, $\|\Phi(f + F^\perp)\|\le \|f + F^\perp\|$.

(ii). First, $\Phi f : E\to\K$, $x\mapsto f(x + F)$, is linear. Since
$$
|(\Phi f)(x)| = |f(x+F)|\le \|f\|\cdot\|x+F\|\le\|f\|\cdot\|x\|,
$$
we have $\Phi f\in E^*$. If $x\in F$, then $(\Phi f)(x) = f(F) = 0$, hence $\Phi f\in F^\perp$. Therefore, indeed $\Phi : (E/F)^*\to F^\perp$. It is obvious that $\Phi$ is linear and injective. To prove surjectivity, let $g\in F^\perp$, and let $f : E/F\to\K$ be defined by
$$
f(x+F) = g(x),\quad x\in E.
$$
The functional $f$ is is well-defined since $F\sse\ker g$. Moreover, $f$ is linear, and for all $x\in E$, $y\in F$ we have
$$
|f(x+F)| = |g(x)| = |g(x+y)|\le\|g\|\cdot\|x+y\|,
$$
which implies $f\in (E/F)^*$. In addition, $\Phi f = g$, and surjectivity is proved.

It remains to show that $\Phi$ is isometric. For this, note that $|(\Phi f)(x)|\le \|f\|\cdot\|x\|$, $x\in E$, implies $\|\Phi f\|\le\|f\|$ for all $f\in (E/F)^*$. On the other hand, for each $\veps > 0$ there exists $x\in E$ with
$$
\|x + F\| = 1\quad\text{and}\quad |f(x + F)|\ge\|f\| - \veps.
$$
Since $1 = \|x + F\| = \inf_{y\in F}\|x + y\|$, there exists $y\in F$ with $\|x + y\|\le 1+\veps$. This implies $\|\frac{x+y}{1+\veps}\|\le 1$ and hence
$$
\left|(\Phi f)\left(\frac{x+y}{1+\veps}\right)\right| = \frac{|f(x+F)|}{1+\veps}\ge\frac{\|f\| - \veps}{1+\veps}.
$$
Thus $\|\Phi f\|\ge \frac{\|f\| - \veps}{1+\veps}$, which yields $\|\Phi f\|\ge\|f\|$. This proves that $\Phi$ is isometric.
\end{proof}

\begin{lemma}
Let $E$ be a normed space over $\K$. For $x\in E$ let $\hat x : E^*\to\K$ be defined by $\hat x(\ell) = \ell(x)$, $\ell\in E^*$. Then $\Lambda_E : E\to (E^*)^*$, $\Lambda_Ex = \hat x$, is an isometric linear operator.
\end{lemma}
\begin{proof}
$\Lambda_E$ is linear, and for $x\in E$ we have
$$
\sup\{|\hat x(\ell)| : \ell\in E^*,\,\|\ell\|=1\} = \sup\{|\ell(x)| : \ell\in E^*,\,\|\ell\|=1\} = \|x\|,
$$
where the last equality follows from Corollary 4.6\marginpar{Reference!}. This proves that indeed $\hat x\in (E^*)^*$ and that $\|\Lambda_E x\| = \|\hat x\| = \|x\|$ for all $x\in E$.
\end{proof}

\begin{definition}
Let $E$ be a normed space over $\K$, and let $\Lambda_E$ be defined as above. Then $\Lambda_E$ is called {\em canonical map} or {\em embedding} of $E$ in $E^{**} := (E^*)^*$. $E$ is {\em reflexive}, if $\Lambda_E$ is surjective. $E^{**}$ is the {\em bi-dual} of $E$, and $\ol{\Lambda_E(E)}$ is the {\em completion} of $E$.
\end{definition}

\begin{rem}
Only Banach spaces can be reflexive. The class of reflexive spaces is a highly important class of Banach spaces. Intriguingly, there exist non-reflexive Banach spaces, which are isometrically isomorphic to their bi-dual. Moreover, notice that finite-dimensional spaces are always reflexive because of $\dim E^{**} = \dim E$.
\end{rem}

\begin{theo}\label{4:t:refl_1}
Let $E$ be a normed space over $\K$.
\begin{enumerate}
\item[(i)] If $E$ is reflexive and $F\sse E$ a closed linear subspace, then $F$ is also reflexive.
\item[(ii)] If $E$ is a Banach space, then $E$ is reflexive if and only if $E^*$ is reflexive.
\end{enumerate}
\end{theo}
\begin{proof}
(i). We have to show that for each $\vphi\in F^{**}$ there exists some $y\in F$ with $\vphi(f) = f(y)$ for all $f\in F^*$. For this, let $\vphi\in F^{**}$ and let $\psi : E^*\to\K$ be defined by $\psi(\ell) := \vphi(\ell|_F)$, $\ell\in E^*$. Since
$$
|\psi(\ell)|\le\|\vphi\|\cdot\|\ell|_F\|\le\|\vphi\|\cdot\|\ell\|,
$$
we have $\psi\in E^{**}$. $E$ being reflexive then implies that there exists $y\in E$ with $\psi(\ell) = \ell(y)$ for all $\ell\in E^*$. Next, towards a contradiction, assume that $y\notin F$. Then there exists some $\ell\in E^*$ with $\ell(y)\neq 0$ and $\ell|_F = 0$. Hence, $0\neq\ell(y) = \psi(\ell) = \vphi(\ell|_F) = 0$. A contradiction. Finally, for $f\in F^*$ there exists some $\ell\in E^*$ with $\ell|_F = f$, hence
$$
\vphi(f) = \vphi(\ell|_F) = \psi(\ell) = \ell(y) = f(y).
$$
This shows that $F$ is reflexive.

(ii). Let $E$ be reflexive. We need to show that for each $u\in E^{***}$ there exists some $f\in E^*$ with $u(\vphi) = \vphi(f)$ for all $\vphi\in E^{**}$. For this, let $u\in E^{***}$, and set $f(x) := u(\hat x)$, $x\in E$. Then $f\in E^*$. Next, let $\vphi\in E^{**}$. Since there hence exists some $x\in E$ with $\hat x = \vphi$, we obtain
$$
u(\vphi) = u(\hat x) = f(x) = \hat x(f) = \vphi(f).
$$
Therefore, $E^*$ is reflexive.

For the converse, let $E^*$ be reflexive. Then, by the above, $E^{**}$ is reflexive. By (i), also $\Lambda_E(E) = \ol{\Lambda_E(E)}$ is reflexive. The claim now follows from the fact that $E$ and $\Lambda_E(E)$ are (isometrically) isomorphic (see Exercise Sheet 6, Exercise 1(ii)).
\end{proof}

\begin{theo}
Let $E$ be a Banach space and $F\sse E$ a closed linear subspace. Then the following are equivalent:
\begin{enumerate}
\item[(i)] $E$ is reflexive.
\item[(ii)] $F$ and $E/F$ are reflexive.
\end{enumerate}
\end{theo}
\begin{proof}
(i)$\Ra$(ii). By (i) and Theorem \ref{4:t:refl_1}(i), also $F$ is reflexive. By Theorem \ref{4:t:refl_1}(ii), $E^*$ is reflexive, hence $F^\perp$ is reflexive. By Theorem \ref{4:t:isoiso}(ii), $F^\perp$ is isometrically isomorphic to $(E/F)^*$ which is therefore also reflexive. By Theorem \ref{4:t:refl_1}(ii), this finally implies that $E/F$ is reflexive.

(ii)$\Ra$(i). Let $\vphi\in E^{**}$. We will again use the isometric isomorphism
$$
\Phi : (E/F)^*\to F^\perp\sse E^*,\quad (\Phi u)(x) = u(x+F),\;u\in (E/F)^*,\,x\in E,
$$
from Theorem \ref{4:t:isoiso}. Then we can define $\psi\in (E/F)^{**}$ by
$$
\psi(u) := \vphi(\Phi u),\quad u\in (E/F)^*.
$$
Since $E/F$ is reflexive, there exists some $x\in E$ with $\wh{x+F} = \psi$, i.e.
$$
\vphi(\Phi u) = \psi(u) = (\wh{x+F})(u) = u(x+F) = (\Phi u)(x) = \hat x(\Phi u),\quad u\in(E/F)^*.
$$
Hence, $(\vphi - \hat x)|_{F^\perp} = 0$.

To utilize the reflexivity of $F$, we next define a suitable $\rho\in F^{**}$. For each $f\in F^*$, choose some $g\in E^*$ with $g|_F = f$ and $\|g\| = \|f\|$. Then define
$$
\rho(f) := (\vphi - \hat x)(g).
$$
This is a proper definition since for two extensions $g,h\in E^*$ of $f$ we have $(g - h)|_F = 0$ and thus $g - h\in F^\perp$. A similar argument shows that $\rho$ is linear. Moreover,
$$
|\rho(f)|\le\|\vphi - \hat x\|\cdot\|g\| = \|\vphi - \hat x\|\cdot\|f\|.
$$
Thus, $\rho\in F^{**}$. As $F$ is reflexive, there exists some $y\in F$ with $\rho(f) = f(y)$ for all $f\in F^*$. Now, we conclude that for all $g\in E^*$ we have
$$
\hat y(g) = g(y) = (g|_F)(y) = \rho(g|_F) = (\vphi - \hat x)(h)
$$
with some $h\in E^*$ satisfying $h|_F = g|_F$ and $\|h\| = \|g|_F\|$. Hence, $h - g\in F^\perp$ and thus $\hat y(g) = (\vphi - \hat x)(g)$ for all $g\in E^*$. Equivalently, $\vphi = \hat x + \hat y = \wh{x + y}\in\Lambda_E(E)$, which shows that $E$ is reflexive.
\end{proof}

\begin{theo}
Let $E$ and $F$ be normed spaces, $E\neq\{0\}$. If $\calL(E,F)$ is complete, then so is $F$.
\end{theo}
\begin{proof}
First, choose $x_0\in E$ with $\|x_0\|=1$. Then there exists some $f\in E^*$ with $f(x_0) = \|x_0\| = 1 = \|f\|$. Next, let $(y_n)_{n\in\N}\sse F$ be a Cauchy sequence, and define $T_n : E\to F$ by
$$
T_nx := f(x)y_n,\quad x\in E.
$$
Since $\|T_nx\|\le\|f\|\|x\|\|y_n\| = \|y_n\|\|x\|$ for $x\in E$, we have $T_n\in\calL(E,F)$. Further, $\|T_n x- T_m x\| = |f(x)|\|y_n - y_m\|\le\|y_n - y_m\|\|x\|$ implies
$$
\|T_n - T_m\|\le\|y_n - y_m\|.
$$
Hence $(T_n)_{n\in\N}$ is a Cauchy sequence in $\calL(E,F)$ and thus converges to some $T\in\calL(E,F)$. This implies $y_n = T_nx_0\to Tx_0$ as $n\to\infty$, i.e.\ $(y_n)_{n\in\N}$ converges in $F$.
\end{proof}

